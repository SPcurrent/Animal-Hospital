\documentclass{jsarticle}

\usepackage{titlesec}
	\titleformat*{\section}{\large\bfseries}
	\titleformat*{\subsection}{\normalsize\bfseries}

\usepackage[format=hang,font=normalsize]{caption}

\usepackage[dvipdfmx]{graphicx}
\makeatletter
	\renewcommand{\thefigure}{\thesection.\arabic{figure}}
	\@addtoreset{figure}{section}
\makeatother

\usepackage[dvipdfmx]{graphicx}
\makeatletter
	\renewcommand{\thetable}{\thesection.\arabic{table}}
	\@addtoreset{table}{section}
\makeatother

\makeatletter
\@addtoreset{equation}{section}
\def\theequation{\thesection.\arabic{equation}}
\makeatother

\usepackage[version=3]{mhchem}
\usepackage{textcomp}
\usepackage{mathcomp}
\usepackage{mathtools,amssymb}
\usepackage{tensor}
\usepackage{exscale}
\usepackage{latexsym}
\usepackage{amsmath}
\usepackage{siunitx}

\usepackage{wrapfig}
\usepackage{float}
\usepackage{subcaption}
\usepackage{booktabs}
\usepackage{array}
\usepackage{multirow}

\usepackage{okumacro}
\usepackage{pxrubrica}
\newcommand{\Slash}[1]{{\looaloign{\hfil/\hfil\crcr$#1$}}}
\usepackage{otf}
\usepackage{pifont}
\usepackage{fancybox}

\usepackage[T1]{fontenc}
\usepackage{lmodern}

\usepackage{url}

\pagestyle{empty}

\renewcommand\kanjifamilydefault{\gtdefault}
\renewcommand\familydefault{\sfdefault}

\usepackage[top=30truemm,bottom=10truemm,left=25truemm,right=25truemm]{geometry}

\begin{document}

\begin{figure}[htbp]
	\begin{center}
		\includegraphics[width=5cm]{Tra2.png}
	\end{center}
\end{figure}

\Huge
\vspace{-10pt}
  猫には避妊去勢手術を!

\Large
  家猫, 外猫を問わず, 猫には避妊去勢手術を受けさせましょう。

  *純血種等で繁殖させたい方は対象外です。


\vspace{30pt}

	\begin{figure}[htbp]
		\centering
		\begin{subfigure}{0.4\columnwidth}
			\centering
			\includegraphics[width=\columnwidth]{2.jpg}
		\end{subfigure}
		\begin{subfigure}{0.4\columnwidth}
			\centering
			\includegraphics[width=\columnwidth]{1.jpg}
		\end{subfigure}
	\end{figure}

\Large
  生後約3か月半で保護された雌猫ちゃん, 保護期間中に3日間脱走して帰ってきたんですが...生後半年を待たずに出産してしまいました。生後半年ですが, 元気に子猫を2ひき育てています。

  *保護期間中とは, 保護してから約1か月のことです。

	このように脱走して数日間, 帰ってこなければ, ほぼ100\%妊娠します。

	\vspace{50pt}
	\begin{wrapfigure}[2]{r}{2cm}
		\vspace*{-\intextsep}
		\includegraphics[width=2cm]{qr.jpg}
	\end{wrapfigure}
	予約問い合わせはこちら(LINE)\Pisymbol{psy}{"AE}

	Gmail:251ousin@gmail.com




\newpage
\huge
  2\si{\kilogram}あれば手術できます!

	\begin{figure}[htbp]
		\centering
		\begin{subfigure}{0.4\columnwidth}
			\centering
			\includegraphics[width=\columnwidth]{3.jpg}
		\end{subfigure}
		\begin{subfigure}{0.4\columnwidth}
			\centering
			\includegraphics[width=\columnwidth]{4.jpg}
		\end{subfigure}
	\end{figure}

\Large
  子猫は痛みや麻酔のリスクが少なく, 傷口も小さいので回復が早いです。術後2-3時間でご飯を食べたり遊んだりする子が多いです。

	避妊去勢手術の

	\begin{figure}[htbp]
		\Large
		\centering
		\begin{minipage}{0.4\columnwidth}
			\ovalbox{メリット}
			\begin{itemize}
				\item 発情時の鳴き声やケンカがなくなる。
				\item オスはスプレー(マーキング)しなくなるので, 尿の臭いが激減します。
				\item 子猫が生まれない。
			\end{itemize}
		\end{minipage}
		\begin{minipage}{0.4\columnwidth}
			\ovalbox{デメリット}
			\begin{itemize}
				\item 費用が掛かる\Pisymbol{psy}{"AE}各種助成制度を利用することで負担を減らすことができます。\\\\
			\end{itemize}
		\end{minipage}
	\end{figure}

	\begin{wrapfigure}[1]{r}{3cm}
		\includegraphics[width=1.4cm]{6.jpg}
		\includegraphics[width=1.4cm]{7.jpg}
	\end{wrapfigure}
	どらにゃんクリニックは神戸市管理票やどうぶつ基金のチケットが使えます。

	\vspace{30pt}
	\begin{wrapfigure}[1]{r}{3cm}
		\vspace*{-\intextsep}
		\includegraphics[width=3cm]{5.jpg}
	\end{wrapfigure}
	来院は手術を受ける猫ちゃんが圧倒的に多いですが, 診察のみの方やワンちゃんも来てくれます。

	\vspace{30pt}
	\begin{wrapfigure}[3]{r}{3cm}
		\vspace*{-\intextsep}
		\includegraphics[width=3cm]{URLdoranyan.png}
	\end{wrapfigure}
	どらにゃんクリニック

	神戸市北区鈴蘭台南町3丁目9-3

	アパルトサンクフォイユ102号室

	URL:\url{https://sunnyday-doranyann.ssl-lolipop.jp}

\end{document}
